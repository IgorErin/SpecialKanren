% !TeX spellcheck = ru_RU
% !TEX root = vkr.tex

\label{sec:relatedworks}

В данном разделе приведен краткий обзор подходов к специализации программ. 

\subsection{Подходы к специализации}
Известны множество подходов к специализации императивных, функциональных и логических языков. Такие методы, как частичные вычисления~\cite{pargen}, суперкомпиляция~\cite{supercompiler}, дистилляция~\cite{distillation} и частичная дедукция~\cite{parded}. 

Все эти подходы объединяет символическое исполнение обрабатываемой программы, называемое driving, в процессе которого строится так называемое processing tree, потенциально бесконечное, что должно отразить саму сущность программы~\cite{supercompiler}. В процессе построения термы, расположенные в узлах дерева подвергаются проверкам, направленным на установление расхождения, например, с помощью~\cite{embedding}. После чего, по полученной структуре генерируется результирующая программа.

Реляционное программирование отлично от логического полнотой поиска~\cite{miniKanren}. В настоящий момент техники специализации, основанные на оных для логических языков, разрабатываются для реляционных программ~\cite{miniDeduction}. 
