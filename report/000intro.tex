% !TeX spellcheck = ru_RU
% !TEX root = vkr.tex 

Реляционное программирование --- это парадигма, основанная на выражении программ с помощью отношений. Отношения сами собой представляют функции, но, в отличие от функционального программирования, исполнять их можно в различных направлениях. Это позволяет естественно выражать некоторые проблемы~\cite{unapp}, среди которых генерация программ посредством написания реляционных интерпретаторов~\cite{relinter}.

Повышение абстракции зачастую приводит к худшей производи-
тельности. Не стали исключением и реляционные языки программирования~\cite{miniKanren, miniDeduction}.

На листинге \ref{le} изображено отношение меньше или равно.

\begin{lstlisting}[caption=Отношение меньше или равно, language=OCaml, frame=single, label = le]
let rec is_le x y is =
    conde
      [ x === O &&& (is === true)
      ; x =/= O &&& (y === O) &&& (false === is)
      ; fresh (x' y') 
          x === S x' 
          &&& (y === S y')
          &&& le x' y' is
      ]
\end{lstlisting}

Иногда пользователя интересуют те случаи, когда некоторые аргументы известны до исполнения. К примеру, в случае упомянутого выше отношения, обнаружить пары чисел, для которых это отношение выполняется.

\begin{lstlisting}[caption=Отношение меньше или равно, language=OCaml, frame=single, label = le]
    let le x y = is_le x y true
\end{lstlisting}

Реляционное программирование предоставляет б\'ольшую гибкость, так как заменив значение одного аргумента позволяет определить обратное отношение.

\begin{lstlisting}[caption=Отношение меньше или равно, language=OCaml, frame=single, label = le]
    let gt x y = is_le x y false
\end{lstlisting}

Возможно, именно эта гибкость становится преградой, что скрывает за собой производительность. Однако, до исполнения можно произвести подстановку определенного аргумента в тело функции, дабы частично вычислить ее, отбросить все лишнее, тем самым получить, возможно, более производительную версию.

Отношения содержащие параметры с конечным доменом, например, имеющие тип \verb|bool|, представляют особенный интерес. Так как в зависимости от того, какое значение будет использовано в качестве аргумента, \verb|true| или \verb|false|, при интерпретации могут быть задействованы совершенно разные части реляционной формулы.

Таким образом, данная работа посвящена исследованию вопроса подстановки и специализации реляционной программы для аргументов с конечным доменом.

\blfootnote{Дата сборки: \today }
