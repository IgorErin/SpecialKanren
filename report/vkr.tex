% !TEX TS-program = xelatex
% !BIB program = bibtex
% !TeX spellcheck = ru_RU

% About magic macroses see also
% https://tex.stackexchange.com/questions/78101/

% По умолчанию используется шрифт 14 размера. Если нужен 12-й шрифт, уберите опцию [14pt]
\documentclass[14pt
  , russian
  %, xcolor={svgnames}
  ]{matmex-diploma-custom}
\usepackage[table]{xcolor}
\usepackage{graphicx}
\usepackage{tabularx}
\newcolumntype{Y}{>{\centering\arraybackslash}X}
\usepackage{amsmath}
\usepackage{amsthm}
\usepackage{amsfonts}
\usepackage{amssymb}
\usepackage{mathtools}
\usepackage{thmtools}
\usepackage{thm-restate}
\usepackage{tikz}
\usepackage{wrapfig}
% \usepackage[kpsewhich,newfloat]{minted}
% \usemintedstyle{vs}
\usepackage[inline]{enumitem}
\usepackage{subcaption}
\usepackage{caption}
%\usepackage[nocompress]{cite}
\usepackage{makecell}
% \setitemize{noitemsep,topsep=0pt,parsep=0pt,partopsep=0pt}
% \setenumerate{noitemsep,topsep=0pt,parsep=0pt,partopsep=0pt}


\graphicspath{ {resources/} }

%
% % \documentclass
% %   [ a4paper        % (Predefined, but who knows...)
% %   , draft,         % Show bad things.
% %   , 12pt           % Font size.
% %   , pagesize,      % Writes the paper size at special areas in DVI or
% %                    % PDF file. Recommended for use.
% %   , parskip=half   % Paragraphs: noindent + gap.
% %   , numbers=enddot % Pointed numbers.
% %   , BCOR=5mm       % Binding size correction.
% %   , submission
% %   , copyright
% %   , creativecommons
% %   ]{eptcs}
% % \providecommand{\event}{ML 2018}  % Name of the event you are submitting to
% % \usepackage{breakurl}             % Not needed if you use pdflatex only.
%
% \usepackage{underscore}           % Only needed if you use pdflatex.
%
% \usepackage{booktabs}
% \usepackage{amssymb}
% \usepackage{amsmath}
% \usepackage{mathrsfs}
% \usepackage{mathtools}
% \usepackage{multirow}
% \usepackage{indentfirst}
% \usepackage{verbatim}
% \usepackage{amsmath, amssymb}
% \usepackage{graphicx}
% \usepackage{xcolor}
% \usepackage{url}
% \usepackage{stmaryrd}
% \usepackage{xspace}
% \usepackage{comment}
% \usepackage{wrapfig}
% \usepackage[caption=false]{subfig}
% \usepackage{placeins}
% \usepackage{tabularx}
% \usepackage{ragged2e}
% \usepackage{soul}
\usepackage{csquotes}
% \usepackage{inconsolata}
%
% \usepackage{polyglossia}   % Babel replacement for XeTeX
%   \setdefaultlanguage[spelling=modern]{russian}
%   \setotherlanguage{english}
% \usepackage{fontspec}    % Provides an automatic and unified interface
%                          % for loading fonts.
% \usepackage{xunicode}    % Generate Unicode chars from accented glyphs.
% \usepackage{xltxtra}     % "Extras" for LaTeX users of XeTeX.
% \usepackage{xecyr}       % Help with Russian.
%
% %% Fonts
% \defaultfontfeatures{Mapping=tex-text}
% \setmainfont{CMU Serif}
% \setsansfont{CMU Sans Serif}
% \setmonofont{CMU Typewriter Text}

\usepackage[final]{listings}

\lstdefinelanguage{ocaml}{
keywords={@type, function, fun, let, in, match, with, when, class, type,
nonrec, object, method, of, rec, repeat, until, while, not, do, done, as, val, inherit, and,
new, module, sig, deriving, datatype, struct, if, then, else, open, private, virtual, include, success, failure,
lazy, assert, true, false, end},
sensitive=true,
commentstyle=\small\itshape\ttfamily,
keywordstyle=\ttfamily\bfseries, %\underbar,
identifierstyle=\ttfamily,
basewidth={0.5em,0.5em},
columns=fixed,
fontadjust=true,
literate={->}{{$\to$}}3 {===}{{$\equiv$}}1 {=/=}{{$\not\equiv$}}1 {|>}{{$\triangleright$}}3 {\\/}{{$\vee$}}2 {/\\}{{$\wedge$}}2 {>=}{{$\ge$}}1 {<=}{{$\le$}} 1,
morecomment=[s]{(*}{*)}
}

\lstset{
mathescape=true,
%basicstyle=\small,
identifierstyle=\ttfamily,
keywordstyle=\bfseries,
commentstyle=\scriptsize\rmfamily,
basewidth={0.5em,0.5em},
fontadjust=true,
language=ocaml
}

\newcommand{\cd}[1]{\texttt{#1}}
\newcommand{\inbr}[1]{\left<#1\right>}


\newcolumntype{L}[1]{>{\raggedright\let\newline\\\arraybackslash\hspace{0pt}}m{#1}}
\newcolumntype{C}[1]{>{\centering\let\newline\\\arraybackslash\hspace{0pt}}m{#1}}
\newcolumntype{R}[1]{>{\raggedleft\let\newline\\\arraybackslash\hspace{0pt}}m{#1}}



\usepackage{soul}
\usepackage[normalem]{ulem}
%\sout{Hello World}

% перевод заголовков в листингах
\renewcommand\lstlistingname{Листинг}
\renewcommand\lstlistlistingname{Листинги}

\newcommand{\vsharp}{\textsc{V$\sharp$}}
\newcommand{\fsharp}{\textsc{F$\sharp$}}
\newcommand{\csharp}{\textsc{C$\sharp$}}

\newcommand{\GitHub}{\textsc{GitHub}}
\newcommand{\SMT}{\textsc{SMT}}

\usepackage{afterpage}
\usepackage{pdflscape}

% swapping \phi and \varphi
% https://tex.stackexchange.com/a/50365/171947
%\expandafter\mathchardef\expandafter\varphi\number\expandafter\phi\expandafter%\relax
%\expandafter\mathchardef\expandafter\phi\number\varphi

%https://tex.stackexchange.com/questions/30720/footnote-without-a-marker
\newcommand\blfootnote[1]{%
	\begingroup
	\renewcommand\thefootnote{}\footnote{#1}%
	\addtocounter{footnote}{-1}%
	\endgroup
}

% TODO: Понять, почему я выделил то, что тут в отдельный файл
\usepackage{listings}
\usepackage{tikz}
\usetikzlibrary{decorations.pathreplacing,calc,shapes,positioning,tikzmark}

\newcounter{tmkcount}

\tikzset{
  use tikzmark/.style={
    remember picture,
    overlay,
    execute at end picture={
      \stepcounter{tmkcount}
    },
  },
  tikzmark suffix={-\thetmkcount}
}


\usepackage{comment}
\usepackage{booktabs}%midrule/toprule/...

\usepackage{siunitx} % для таблиц с едлиницами измерений



\usepackage{totcount}
\usepackage{placeins}

\usepackage{pseudocode}
\usepackage{caption}
\usepackage{listings}

\usepackage{algorithm,algpseudocode}
\usepackage{amsmath}

\DeclareCaptionFont{white}{ \color{white} }
\DeclareCaptionFormat{listing}{
    \parbox{\textwidth}{\hspace{15pt}#1#2#3}
}
\captionsetup[lstlisting]{ format=listing
  %, labelfont=white, textfont=white
  , singlelinecheck=false, margin=0pt, font={bf}
}

\begin{document}
% !TeX spellcheck = ru_RU
% !TEX root = vkr.tex

%% Если что-то забыли, при компиляции будут ошибки Undefined control sequence \my@title@<что забыли>@ru
%% Если англоязычная титульная страница не нужна, то ее можно просто удалить.
\filltitle{ru}{
    %% Актуально только для курсовых/практик. ВКР защищаются не на кафедре а в ГЭК по направлению, 
    %%   и к моменту защиты вы будете уже не в группе.
    chair              = {Кафедра системного программирования},
    group              = {21.Б10-мм},
    %
    %% Макрос filltitle ненавидит пустые строки, поэтому обязателен хотя бы символ комментария на строке
    %% Актуально всем.
    title              = {Специализация реляционных программ},
    % 
    %% Здесь указывается тип работы. Возможные значения:
    %%   coursework - отчёт по курсовой работе;
    %%   practice - отчёт по учебной практике;
    %%   prediploma - отчёт по преддипломной практике;
    %%   master - ВКР магистра;
    %%   bachelor - ВКР бакалавра.
    type               = {practice},
    %
    %% Здесь указывается вид работы. От вида работы зависят критерии оценивания.
    %%   solution - <<Решение>>. Обучающемуся поручили найти способ решения проблемы в области разработки программного обеспечения или теоретической информатики с учётом набора ограничений.
    %%   experiment - <<Эксперимент>>. Обучающемуся поручили изучить возможности, достоинства и недостатки новой технологии, платформы, языка и т. д. на примере какой-то задачи.
    %%   production - <<Производственное задание>>. Автору поручили реализовать потенциально полезное программное обеспечение.
    %%   comparison - <<Сравнение>>. Обучающемуся поручили сравнить несколько существующих продуктов и/или подходов.
    %%   theoretical - <<Теоретическое исследование>>. Автору поручили доказать какое-то утверждение, исследовать свойства алгоритма и т.п., при этом не требуя написания кода.
    kind               = {production},
    %
    author             = {Ерин Игорь Антонович},
    % 
    %% Актуально только для ВКР. Указывается код и название направления подготовки. Типичные примеры:
    %%   02.03.03 <<Математическое обеспечение и администрирование информационных систем>>
    %%   02.04.03 <<Математическое обеспечение и администрирование информационных систем>>
    %%   09.03.04 <<Программная инженерия>>
    %%   09.04.04 <<Программная инженерия>>
    %% Те, что с 03 в середине --- бакалавриат, с 04 --- магистратура.
    specialty          = {02.03.03 <<Математическое обеспечение и администрирование информационных систем>>},
    % 
    %% Актуально только для ВКР. Указывается шифр и название образовательной программы. Типичные примеры:
    %%   СВ.5006.2017 <<Математическое обеспечение и администрирование информационных систем>>
    %%   СВ.5162.2020 <<Технологии программирования>>
    %%   СВ.5080.2017 <<Программная инженерия>>
    %%   ВМ.5665.2019 <<Математическое обеспечение и администрирование информационных систем>>
    %%   ВМ.5666.2019 <<Программная инженерия>>
    %% Шифр и название программы можно посмотреть в учебном плане, по которому вы учитесь. 
    %% СВ.* --- бакалавриат, ВМ.* --- магистратура. В конце --- год поступления (не обязательно ваш, если вы были в академе/вылетали).
    programme          = {СВ.5006.2019 <<Математическое обеспечение и администрирование информационных систем>>},
    % 
    %% Актуально только для ВКР, только для матобеса и только 2017-2018 годов поступления. Указывается профиль подготовки, на котором вы учитесь.
    %% Названия профилей можно найти в учебном плане в списке дисциплин по выбору. На каком именно вы, вам должны были сказать после второго курса (можно уточнить в студотделе).
    %% Вот возможные вариканты:
    %%   Математические основы информатики
    %%   Информационные системы и базы данных
    %%   Параллельное программирование
    %%   Системное программирование
    %%   Технология программирования
    %%   Администрирование информационных систем
    %%   Реинжиниринг программного обеспечения
    % profile            = {Системное программирование},
    % 
    %% Актуально всем.
    %supervisorPosition = {проф. каф. СП, д.ф.-м.н., проф.}, % Терехов А.Н.
    supervisorPosition = {ассистент кафедры системного программирования,}, % Григорьев С.В.
    supervisor         = {Косарев Д. С.},
    % 
    %% Актуально только для практик и курсовых. Если консультанта нет, закомментировать или удалить вовсе.
    %consultantPosition = {должность ООО <<Место работы>> степень},
    %consultant         = {К.~К.~Консультант},
    %
    %% Актуально только для ВКР.
    %reviewerPosition   = {должность ООО <<Место работы>> степень},
    %reviewer           = {Р.~Р.~Рецензент},
}

% \filltitle{en}{
%     chair              = {Advisor's chair},
%     group              = {ХХ.BХХ-mm},
%     title              = {Template for SPbU qualification works},
%     type               = {practice},
%     author             = {FirstName Surname},
%     % 
%     %% Possible choices:
%     %%   02.03.03 <<Software and Administration of Information Systems>>
%     %%   02.04.03 <<Software and Administration of Information Systems>>
%     %%   09.03.04 <<Software Engineering>>
%     %%   09.04.04 <<Software Engineering>>
%     %% Те, что с 03 в середине --- бакалавриат, с 04 --- магистратура.
%     specialty          = {02.03.03 ``Software and Administration of Information Systems''},
%     % 
%     %% Possible choices:
%     %%   СВ.5006.2017 <<Software and Administration of Information Systems>>
%     %%   СВ.5162.2020 <<Programming Technologies>>
%     %%   СВ.5080.2017 <<Software Engineering>>
%     %%   ВМ.5665.2019 <<Software and Administration of Information Systems>>
%     %%   ВМ.5666.2019 <<Software Engineering>>
%     programme          = {СВ.5006.2019 ``Software and Administration of Information Systems''},
%     % 
%     %% Possible choices:
%     %%   Mathematical Foundations of Informatics
%     %%   Information Systems and Databases
%     %%   Parallel Programming
%     %%   System Programming
%     %%   Programming Technology
%     %%   Information Systems Administration
%     %%   Software Reengineering
%     % profile            = {Software Engineering},
%     % 
%     %% Note that common title translations are:
%     %%   кандидат наук --- C.Sc. (NOT Ph.D.)
%     %%   доктор ... наук --- Sc.D.
%     %%   доцент --- docent (NOT assistant/associate prof.)
%     %%   профессор --- prof.
%     supervisorPosition = {Sc.D, prof.},
%     supervisor         = {S.S. Supervisor},
%     % 
%     consultantPosition = {position at ``Company'', degree if present},
%     consultant         = {C.C. Consultant},
%     %
%     reviewerPosition   = {position at ``Company'', degree if present},
%     reviewer           = {R.R. Reviewer},
% }
\maketitle
\setcounter{tocdepth}{2}
\tableofcontents

% \begin{abstract}
%   В курсаче не нужен
% \end{abstract}

\section*{Введение}
% !TeX spellcheck = ru_RU
% !TEX root = vkr.tex 

Реляционное программирование --- это парадигма, основанная на выражении программ с помощью отношений. Отношения сами собой представляют функции, но, в отличие от функционального программирования, исполнять их можно в различных направлениях. Это позволяет естественно выражать некоторые проблемы~\cite{unapp}, среди которых генерация программ посредством написания реляционных интерпретаторов~\cite{relinter}.

Повышение абстракции зачастую приводит к худшей производи-
тельности. Не стали исключением и реляционные языки программирования~\cite{miniKanren, miniDeduction}.

На листинге \ref{le} изображено отношение меньше или равно.

\begin{lstlisting}[caption=Отношение меньше или равно, language=OCaml, frame=single, label = le]
let rec is_le x y is =
    conde
      [ x === O &&& (is === true)
      ; x =/= O &&& (y === O) &&& (false === is)
      ; fresh (x' y') 
          x === S x' 
          &&& (y === S y')
          &&& le x' y' is
      ]
\end{lstlisting}

Иногда пользователя интересуют те случаи, когда некоторые аргументы известны до исполнения. К примеру, в случае упомянутого выше отношения, обнаружить пары чисел, для которых это отношение выполняется.

\begin{lstlisting}[caption=Отношение меньше или равно, language=OCaml, frame=single, label = le]
    let le x y = is_le x y true
\end{lstlisting}

Реляционное программирование предоставляет б\'ольшую гибкость, так как заменив значение одного аргумента позволяет определить обратное отношение.

\begin{lstlisting}[caption=Отношение меньше или равно, language=OCaml, frame=single, label = le]
    let gt x y = is_le x y false
\end{lstlisting}

Возможно, именно эта гибкость становится преградой, что скрывает за собой производительность. Однако, до исполнения можно произвести подстановку определенного аргумента в тело функции, дабы частично вычислить ее, отбросить все лишнее, тем самым получить, возможно, более производительную версию.

Отношения содержащие параметры с конечным доменом, например, имеющие тип \verb|bool|, представляют особенный интерес. Так как в зависимости от того, какое значение будет использовано в качестве аргумента, \verb|true| или \verb|false|, при интерпретации могут быть задействованы совершенно разные части реляционной формулы.

Таким образом, данная работа посвящена исследованию вопроса подстановки и специализации реляционной программы для аргументов с конечным доменом.

\blfootnote{Дата сборки: \today }


\section{Постановка задачи}
% !TeX spellcheck = ru_RU
% !TEX root = vkr.tex

\label{sec:task}
 Целью данной работы является реализация специализатора реляционных программ написанных на языке OCanren\footnote{Репозиторий проекта OCanren: \url{https://github.com/PLTools/OCanren} \\
(Дата обращения: 10.12.2023)}, диалекте miniKanren\footnote{Сайт языка miniKanren: \url{http://minikanren.org/} 
(Дата обращения: 10.12.2023)}. Для этого были поставлены следующие задачи.
 \begin{itemize}
 \item Реализовать специализатор.
 \item Сравнить производительность специализированных и исходных функций.
 \end{itemize}


\section{Обзор}
% !TeX spellcheck = ru_RU
% !TEX root = vkr.tex

\label{sec:relatedworks}

В данном разделе приведен краткий обзор подходов к специализации программ. 

\subsection{Подходы к специализации}
Известны множество подходов к специализации императивных, функциональных и логических языков. Такие методы, как частичные вычисления~\cite{pargen}, суперкомпиляция~\cite{supercompiler}, дистилляция~\cite{distillation} и частичная дедукция~\cite{parded}. 

Все эти подходы объединяет символическое исполнение обрабатываемой программы, называемое driving, в процессе которого строится так называемое processing tree, потенциально бесконечное, что должно отразить саму сущность программы~\cite{supercompiler}. В процессе построения термы, расположенные в узлах дерева подвергаются проверкам, направленным на установление расхождения, например, с помощью~\cite{embedding}. После чего, по полученной структуре генерируется результирующая программа.

Реляционное программирование отлично от логического полнотой поиска~\cite{miniKanren}. В настоящий момент техники специализации, основанные на оных для логических языков, разрабатываются для реляционных программ~\cite{miniDeduction}. 


\section{Реализация}
% !TeX spellcheck = ru_RU
% !TEX root = vkr.tex

В данном разделе описаны подходы к реализации.

\subsection{Исходное представление}
Так как специализируемым языком выступал OCanren, встроенный в OCaml, в качестве исходного представления программ на стадии проектирования предполагалось использовать одно из представлений, которое порождает компилятор OCaml. Таковым было выбрано типизированное дерево, ибо типы необходимы для установления
конечности домена и генерации всех возможных значений.

\subsection{Промежуточное представление}

В качестве промежуточного представления была выбрана дизъюнктивная нормальная форма (далее ДНФ). Ибо она позволяет рассматривать каждый конъюнкт независимо от других. Что в свою очередь упрощает протягивание констант и редукцию всей формулы.

\subsection{Специализация}
 Рассмотрим редукцию следующей формулы.

\begin{lstlisting}[caption=Отношение вычитания, language=OCaml, frame=single, label = sub]
 let sub x y z =
    fresh (valid)
      loe y x valid
      &&& conde
            [ valid === false &&& (z === None)
            ; fresh (z_value)
                valid === true 
                &&& (z === Some z_value) 
                &&& add y z_value x)
            ]
\end{lstlisting}

Специализация будет происходить по параметру \verb|z| и конструктору \verb|Some|. Необходимо помнить, что арность данного конструктора равна единице.

Сначала формула будет приведена в дизъюнктивную нормальную форму. Объявление свежих переменных необходимо переместить, чтобы их область видимости состояла из всего конъюнкта, при необходимости переименовав. 

\begin{lstlisting}[caption=Отношение в ДНФ, language=OCaml, frame=single, label = sub]
 let sub x y z =
    fresh (valid) (loe y x valid) (valid === false) (z === None)
    ||| fresh (valid z_value)  
          (loe y x valid)
          (valid === true) 
          (z === Some z_value) 
          (add y z_value x)
        
\end{lstlisting}

Редуцируемый параметр будет заменен на параметры конструктора, в данном случае \verb|some_arg|. Все вхождения редуцируемого параметра будут заменены на конструктор с новым параметром в качестве аргумента.

\begin{lstlisting}[caption=Отношение после подстановки конструктора, language=OCaml, frame=single, label = sub]
 let sub x y some_arg =
    fresh (valid) 
        (loe y x valid)
        (valid === false) 
        (Some some_arg === None)
    ||| fresh (valid z_value)  
          (loe y x valid)
          (valid === true) 
          (Some some_arg === Some z_value) 
          (add y z_value x)
\end{lstlisting}

Теперь необходимо протянуть константы.

\subsubsection{Протягивание констант}

Все имена, к которым применялось отношение унификации будут разбиты на классы эквивалентности. После чего  в каждом классе эквивалентности будет выбран представитель. Затем все вхождения свободных переменных этого класса будут заменены на данного представителя.

\begin{lstlisting}[caption=Отношение после протягивания констант, language=OCaml, frame=single, label = sub]
 let sub x y some_arg =
    fresh (valid) 
        (loe y x false)
        (false === false) 
        (Some some_arg === None)
    ||| fresh (valid z_value)  
          (loe y x true)
          (true === true) 
          (Some some_arg === Some z_value) 
          (add y z_value x)
\end{lstlisting}

\subsubsection{Редукция}

Во время редукции производится проверка на выполнимость. Конъюнкты содержащие заведомо невыполнимую унификацию будут удалены. Избыточные конструкторы будут сняты. 

\begin{lstlisting}[caption=Отношение после редукции, language=OCaml, frame=single, label = sub]
 let sub x y some_arg =
    fresh (valid z_value)  
          (loe y x true)
          (some_arg === z_value) 
          (add y z_value x)
\end{lstlisting}

Протягивание констант и редукцию необходимо повторять до схождения к неподвижной точке.

\newpage

\begin{lstlisting}[caption=Отношение с лишними свежими именами, language=OCaml, frame=single, label = sub]
 let sub x y some_arg =
    fresh (valid z_value)  (loe y x true) (add y some_arg x)
\end{lstlisting}

После чего необходимо в каждом конъюнкте независимо удалить неиспользуемые свежие переменные.

\begin{lstlisting}[caption=Результирующее отношение, language=OCaml, frame=single, label = sub]
 let sub x y some_arg =
    fresh ()  (loe y x true) (add y some_arg x)
\end{lstlisting}

\subsubsection{Замыкание}

В результате может получиться формула, в теле которой окажутся вызовы отношений с известными параметрами. Такие отношения так же необходимо специализировать, их специализированный вызовы подставить в текущую формулу. Осуществляется это с помощью обхода графа вызовов, начиная с исходной функции.

\begin{lstlisting}[caption=Отношение со специализированным вызовом, language=OCaml, frame=single, label = sub]
 let sub x y some_arg =
    fresh ()  (loe_true y x) (add y some_arg x)
\end{lstlisting}

\subsection{Трансляция}

По редуцированной формуле строится нетипизированное дерево. Затем средствами стандартной библиотеки компилятора оно транслируется в код языка OCaml.


\section{Эксперимент}
% !TeX spellcheck = ru_RU
% !TEX root = vkr.tex

\newcommand{\NA}{---}

В данном разделе приведены условия сравнительных экспериментов и их результаты.

\subsection{Цель эксперимента}

Целью эксперимента является сравнение производительности специализированных и исходных функций. 

\subsection{Условия эксперимента}

Для эксперимента были выбраны отношения: 
is\_even\footnote{Код отношения is\_even 
\url{https://github.com/IgorErin/SpecialKanren/blob/master/samples/is_even.ml} (Дата обращения 10.12.2023)}, sub\footnote{Код отношения sub : \url{https://github.com/IgorErin/SpecialKanren/blob/master/samples/sub.ml} (Дата обращения 10.12.2023)}, gcw\footnote{Код отношения gcw: \url{https://github.com/IgorErin/SpecialKanren/blob/master/samples/gcw.ml} (Дата обращения 10.12.2023)}, bridge\footnote{Исходный код отношения bridge \url{https://github.com/IgorErin/SpecialKanren/blob/master/samples/bridge.ml} (Дата обращения 10.12.2023)}.

Эксперименты проведены на машине, имеющей следующие характеристики: Ubuntu 20.4, AMD Ryzen 5 5500U, 4.4GHz, DDR4 16GB RAM. Для измерения был использован ocaml-benchmark\footnote{Репозиторий ocaml-benchmark: \url{https://github.com/Chris00/ocaml-benchmark} \\ (Дата обращение 10.12.2023)}.

\subsection{Результаты эксперимента}

Результаты измерений приведены в следующих таблицах, где \verb|x| и \verb|spec_x| обозначают соответственно результаты исполнения функции с конструктором \verb|x| и исполнение специализированной по этому конструктору функции. В столбце \verb|Частота| обозначено количество исполнений за секунду. Большее --- лучше. Во всех случаях отклонение составило менее 3\%, вследствие чего оно не приводится. Каждый эксперимент состоял из 30 замеров.

\begin{table}[H]
\begin{center}
\caption{Результаты измерений is\_even при получении первых ста ответов.} 
    \newcolumntype{P}[1]{>{\centering\arraybackslash}p{#1}}
    \begin{tabular}{|l|C{2cm}|C{2.7cm}|C{2.7cm}|}
    \hline
    Название & Частота & false & spec\_false  \\
    \hline
    \hline
    \rowcolor{black!10} false      &  485  & -- & -30\% \\
    \rowcolor{black!2}  spec\_false &  695   & 43\% & -- \\
    \hline
    \end{tabular}
    \label{evenfalse}
\end{center}
\end{table}

\begin{table}[H]
\begin{center}
\caption{Результаты измерений is\_even при получении первых ста ответов.} 
    \newcolumntype{P}[1]{>{\centering\arraybackslash}p{#1}}
    \begin{tabular}{|l|C{2cm}|C{2.7cm}|C{2.7cm}|}
    \hline
    Название & Частота & true & spec\_true  \\
    \hline
    \hline
    \rowcolor{black!10} true      &  486  & -- & -29\% \\
    \rowcolor{black!2}  spec\_true &  688   & 42\% & -- \\
    \hline
    \end{tabular}
    \label{eventrue}
\end{center}
\end{table}

\begin{table}[H]
\begin{center}
\caption{Результаты измерений sub при получении первых двадцати пяти ответов.} 
    \newcolumntype{P}[1]{>{\centering\arraybackslash}p{#1}}
    \begin{tabular}{|l|C{2cm}|C{2.7cm}|C{2.7cm}|}
    \hline
    Название & Частота & none & spec\_none  \\
    \hline
    \hline
    \rowcolor{black!10} none      &  5598  & -- & -24\% \\
    \rowcolor{black!2}  spec\_none &  7329   & 31\% & -- \\
    \hline
    \end{tabular}
    \label{subnone}
\end{center}
\end{table}

\begin{table}[H]
\begin{center}
\caption{Результаты измерений sub при получении первых двадцати пяти ответов. } 
    \newcolumntype{P}[1]{>{\centering\arraybackslash}p{#1}}
    \begin{tabular}{|l|C{2cm}|C{2.7cm}|C{2.7cm}|}
    \hline
    Название & Частота & some & spec\_some  \\
    \hline
    \hline
    \rowcolor{black!10} some      &  1071  & -- & -4\% \\
    \rowcolor{black!2}  spec\_some &  1113   & 4\% & -- \\
    \hline
    \end{tabular}
    \label{subsome}
\end{center}
\end{table}

\begin{table}[H]
\begin{center}
\caption{Результаты измерений gcw при получение первых ста ответов.} 
    \newcolumntype{P}[1]{>{\centering\arraybackslash}p{#1}}
    \begin{tabular}{|l|C{2cm}|C{2.7cm}|C{2.7cm}|}
    \hline
    Нзавние & Частота &  spec\_false & false  \\
    \hline
    \hline
    \rowcolor{black!10} spec\_false &  366  & -- & -19\% \\
    \rowcolor{black!2}  false      &  454   & 24\% & -- \\
    \hline
    \end{tabular}
    \label{gcwfalse}
\end{center}
\end{table}

\begin{table}[H]
\begin{center}
\caption{Результаты измерений gcw при получении первых ста ответов.} 
    \newcolumntype{P}[1]{>{\centering\arraybackslash}p{#1}}
    \begin{tabular}{|l|C{2cm}|C{2.7cm}|C{2.7cm}|}
    \hline
    Название & Частота & true & spec\_true  \\
    \hline
    \hline
    \rowcolor{black!10} true      &  1.25 & -- & -52\% \\
    \rowcolor{black!2}  spec\_true &  2.61 & 108\% & -- \\
    \hline
    \end{tabular}
    \label{gcwtrue}
\end{center}
\end{table}

\begin{table}[H]
\begin{center}
\caption{Результаты измерений bridge при получении первых ста ответов.} 
    \newcolumntype{P}[1]{>{\centering\arraybackslash}p{#1}}
    \begin{tabular}{|l|C{2cm}|C{2.7cm}|C{2.7cm}|}
    \hline
    Название & Частота &  none & spec\_none  \\
    \hline
    \hline
    \rowcolor{black!10} none      &  421  & -- & -29\% \\
    \rowcolor{black!2}  spec\_none &  596   & 42\% & -- \\
    \hline
    \end{tabular}
    \label{brdigenone}
\end{center}
\end{table}

Ввиду продолжительности исполнения отношения bridge, приведена величина, обратная \verb|Частоте|. То есть в столбце \verb|Скорость| обозначено среднее количество секунд за исполнение. Меньшее --- лучше.

\begin{table}[H]
\begin{center}
\caption{Результаты измерений bridge при получении первого ответа.} 
    \newcolumntype{P}[1]{>{\centering\arraybackslash}p{#1}}
    \begin{tabular}{|l|C{2.3cm}|C{2.7cm}|C{2.7cm}|}
    \hline
    Нзавние & Скорость & some & spec\_some  \\
    \hline
    \hline
    \rowcolor{black!10} some      &  15.8 & -- & -43\% \\
    \rowcolor{black!2}  spec\_some &  9.03 & 75\% & -- \\
    \hline
    \end{tabular}
    \label{brdigesome}
\end{center}
\end{table}

\FloatBarrier

\subsection{Выводы}

Из приведенных результатов видно, что специализированные версии эффективнее всегда, за исключением эксперимента \ref{gcwfalse}, где проигрыш может быть объяснен большим количеством дублицированного кода вследствие использования ДНФ. Во многих конъюнктах создаются большое количество свежих имен, которые в исходной формуле аллоцировались единожды. Данный случай показывает необходимость слияния одинаковых частей конъюнктов, по крайней мере аллокации переменных.

\section*{Заключение}
% !TeX spellcheck = ru_RU
% !TEX root = vkr.tex

В ходе данной работы были выполнены следующие задачи.

\begin{itemize}
    \item Реализован простой специализатор реляционных программ.
    
    Код работы доступен в репозитории на сервисе GitHub\footnote{Репозиторий проекта OCanren: \url{https://github.com/IgorErin/SpecialKanren} \\
    (Дата обращения: 10.12.2023)}. Имя
    пользователя: IgorErin. 
    
    \item Произведено сравнение изменения производительности некоторых функций до специализации и после.
\end{itemize}


Даже такая простая специализация способна значительно улучшить время исполнения реляционных программ.

Далее предполагается:

\begin{itemize}
    \item Увеличить класс обрабатываемых программ. Добавить поддержку модулей, частичного применения и тому подобного.
    \item Улучшить качество генерируемого кода. Уменьшить количество дублицируемых конструкций, поддержать инлайнинг простых реляций.
\end{itemize}

После чего необходимо будет провести эксперимент на настоящих программных продуктах, активно использующих реляционное программирование, например~\cite{JGS}.

\setmonofont[Mapping=tex-text]{CMU Typewriter Text}
  \bibliographystyle{ugost2008ls}
  \bibliography{vkr}
\end{document}
