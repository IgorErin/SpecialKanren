% !TEX TS-program = xelatex
% !BIB program = bibtex
% !TeX spellcheck = ru_RU

% About magic macroses see also
% https://tex.stackexchange.com/questions/78101/

\input{header.tex}
\input{header2.tex}

\usepackage{totcount}
\usepackage{placeins}

\usepackage{pseudocode}
\usepackage{caption}
\usepackage{listings}

\usepackage{algorithm,algpseudocode}
\usepackage{amsmath}

\DeclareCaptionFont{white}{ \color{white} }
\DeclareCaptionFormat{listing}{
    \parbox{\textwidth}{\hspace{15pt}#1#2#3}
}
\captionsetup[lstlisting]{ format=listing
  %, labelfont=white, textfont=white
  , singlelinecheck=false, margin=0pt, font={bf}
}

\begin{document}
\input{title.tex}
\maketitle
\setcounter{tocdepth}{2}
\tableofcontents

% \begin{abstract}
%   В курсаче не нужен
% \end{abstract}

\section*{Введение}
% !TeX spellcheck = ru_RU
% !TEX root = vkr.tex 

Реляционное программирование --- это парадигма, основанная на выражении программ с помощью отношений. Отношения сами собой представляют функции, но, в отличие от функционального программирования, исполнять их можно в различных направлениях. Это позволяет естественно выражать некоторые проблемы~\cite{unapp}, среди которых генерация программ посредством написания реляционных интерпретаторов~\cite{relinter}.

Повышение абстракции зачастую приводит к худшей производи-
тельности. Не стали исключением и реляционные языки программирования~\cite{miniKanren, miniDeduction}.

На листинге \ref{le} изображено отношение меньше или равно.

\begin{lstlisting}[caption=Отношение меньше или равно, language=OCaml, frame=single, label = le]
let rec is_le x y is =
    conde
      [ x === O &&& (is === true)
      ; x =/= O &&& (y === O) &&& (false === is)
      ; fresh (x' y') 
          x === S x' 
          &&& (y === S y')
          &&& le x' y' is
      ]
\end{lstlisting}

Иногда пользователя интересуют те случаи, когда некоторые аргументы известны до исполнения. К примеру, в случае упомянутого выше отношения, обнаружить пары чисел, для которых это отношение выполняется.

\begin{lstlisting}[caption=Отношение меньше или равно, language=OCaml, frame=single, label = le]
    let le x y = is_le x y true
\end{lstlisting}

Реляционное программирование предоставляет б\'ольшую гибкость, так как заменив значение одного аргумента позволяет определить обратное отношение.

\begin{lstlisting}[caption=Отношение меньше или равно, language=OCaml, frame=single, label = le]
    let gt x y = is_le x y false
\end{lstlisting}

Возможно, именно эта гибкость становится преградой, что скрывает за собой производительность. Однако, до исполнения можно произвести подстановку определенного аргумента в тело функции, дабы частично вычислить ее, отбросить все лишнее, тем самым получить, возможно, более производительную версию.

Отношения содержащие параметры с конечным доменом, например, имеющие тип \verb|bool|, представляют особенный интерес. Так как в зависимости от того, какое значение будет использовано в качестве аргумента, \verb|true| или \verb|false|, при интерпретации могут быть задействованы совершенно разные части реляционной формулы.

Таким образом, данная работа посвящена исследованию вопроса подстановки и специализации реляционной программы для аргументов с конечным доменом.

\blfootnote{Дата сборки: \today }


\section{Постановка задачи}
% !TeX spellcheck = ru_RU
% !TEX root = vkr.tex

\label{sec:task}
 Целью данной работы является реализация специализатора реляционных программ написанных на языке OCanren\footnote{Репозиторий проекта OCanren: \url{https://github.com/PLTools/OCanren} \\
(Дата обращения: 10.12.2023)}, диалекте miniKanren\footnote{Сайт языка miniKanren: \url{http://minikanren.org/} 
(Дата обращения: 10.12.2023)}. Для этого были поставлены следующие задачи.
 \begin{itemize}
 \item Реализовать специализатор.
 \item Сравнить производительность специализированных и исходных функций.
 \end{itemize}


\section{Обзор}
% !TeX spellcheck = ru_RU
% !TEX root = vkr.tex

\label{sec:relatedworks}

В данном разделе приведен краткий обзор подходов к специализации программ. 

\subsection{Подходы к специализации}
Известны множество подходов к специализации императивных, функциональных и логических языков. Такие методы, как частичные вычисления~\cite{pargen}, суперкомпиляция~\cite{supercompiler}, дистилляция~\cite{distillation} и частичная дедукция~\cite{parded}. 

Все эти подходы объединяет символическое исполнение обрабатываемой программы, называемое driving, в процессе которого строится так называемое processing tree, потенциально бесконечное, что должно отразить саму сущность программы~\cite{supercompiler}. В процессе построения термы, расположенные в узлах дерева подвергаются проверкам, направленным на установление расхождения, например, с помощью~\cite{embedding}. После чего, по полученной структуре генерируется результирующая программа.

Реляционное программирование отлично от логического полнотой поиска~\cite{miniKanren}. В настоящий момент техники специализации, основанные на оных для логических языков, разрабатываются для реляционных программ~\cite{miniDeduction}. 


\section{Реализация}
% !TeX spellcheck = ru_RU
% !TEX root = vkr.tex

В данном разделе описаны подходы к реализации.

\subsection{Исходное представление}
Так как специализируемым языком выступал OCanren, встроенный в OCaml, в качестве исходного представления программ на стадии проектирования предполагалось использовать одно из представлений, которое порождает компилятор OCaml. Таковым было выбрано типизированное дерево, ибо типы необходимы для установления
конечности домена и генерации всех возможных значений.

\subsection{Промежуточное представление}

В качестве промежуточного представления была выбрана дизъюнктивная нормальная форма (далее ДНФ). Ибо она позволяет рассматривать каждый конъюнкт независимо от других. Что в свою очередь упрощает протягивание констант и редукцию всей формулы.

\subsection{Специализация}
 Рассмотрим редукцию следующей формулы.

\begin{lstlisting}[caption=Отношение вычитания, language=OCaml, frame=single, label = sub]
 let sub x y z =
    fresh (valid)
      loe y x valid
      &&& conde
            [ valid === false &&& (z === None)
            ; fresh (z_value)
                valid === true 
                &&& (z === Some z_value) 
                &&& add y z_value x)
            ]
\end{lstlisting}

Специализация будет происходить по параметру \verb|z| и конструктору \verb|Some|. Необходимо помнить, что арность данного конструктора равна единице.

Сначала формула будет приведена в дизъюнктивную нормальную форму. Объявление свежих переменных необходимо переместить, чтобы их область видимости состояла из всего конъюнкта, при необходимости переименовав. 

\begin{lstlisting}[caption=Отношение в ДНФ, language=OCaml, frame=single, label = sub]
 let sub x y z =
    fresh (valid) (loe y x valid) (valid === false) (z === None)
    ||| fresh (valid z_value)  
          (loe y x valid)
          (valid === true) 
          (z === Some z_value) 
          (add y z_value x)
        
\end{lstlisting}

Редуцируемый параметр будет заменен на параметры конструктора, в данном случае \verb|some_arg|. Все вхождения редуцируемого параметра будут заменены на конструктор с новым параметром в качестве аргумента.

\begin{lstlisting}[caption=Отношение после подстановки конструктора, language=OCaml, frame=single, label = sub]
 let sub x y some_arg =
    fresh (valid) 
        (loe y x valid)
        (valid === false) 
        (Some some_arg === None)
    ||| fresh (valid z_value)  
          (loe y x valid)
          (valid === true) 
          (Some some_arg === Some z_value) 
          (add y z_value x)
\end{lstlisting}

Теперь необходимо протянуть константы.

\subsubsection{Протягивание констант}

Все имена, к которым применялось отношение унификации будут разбиты на классы эквивалентности. После чего  в каждом классе эквивалентности будет выбран представитель. Затем все вхождения свободных переменных этого класса будут заменены на данного представителя.

\begin{lstlisting}[caption=Отношение после протягивания констант, language=OCaml, frame=single, label = sub]
 let sub x y some_arg =
    fresh (valid) 
        (loe y x false)
        (false === false) 
        (Some some_arg === None)
    ||| fresh (valid z_value)  
          (loe y x true)
          (true === true) 
          (Some some_arg === Some z_value) 
          (add y z_value x)
\end{lstlisting}

\subsubsection{Редукция}

Во время редукции производится проверка на выполнимость. Конъюнкты содержащие заведомо невыполнимую унификацию будут удалены. Избыточные конструкторы будут сняты. 

\begin{lstlisting}[caption=Отношение после редукции, language=OCaml, frame=single, label = sub]
 let sub x y some_arg =
    fresh (valid z_value)  
          (loe y x true)
          (some_arg === z_value) 
          (add y z_value x)
\end{lstlisting}

Протягивание констант и редукцию необходимо повторять до схождения к неподвижной точке.

\newpage

\begin{lstlisting}[caption=Отношение с лишними свежими именами, language=OCaml, frame=single, label = sub]
 let sub x y some_arg =
    fresh (valid z_value)  (loe y x true) (add y some_arg x)
\end{lstlisting}

После чего необходимо в каждом конъюнкте независимо удалить неиспользуемые свежие переменные.

\begin{lstlisting}[caption=Результирующее отношение, language=OCaml, frame=single, label = sub]
 let sub x y some_arg =
    fresh ()  (loe y x true) (add y some_arg x)
\end{lstlisting}

\subsubsection{Замыкание}

В результате может получиться формула, в теле которой окажутся вызовы отношений с известными параметрами. Такие отношения так же необходимо специализировать, их специализированный вызовы подставить в текущую формулу. Осуществляется это с помощью обхода графа вызовов, начиная с исходной функции.

\begin{lstlisting}[caption=Отношение со специализированным вызовом, language=OCaml, frame=single, label = sub]
 let sub x y some_arg =
    fresh ()  (loe_true y x) (add y some_arg x)
\end{lstlisting}

\subsection{Трансляция}

По редуцированной формуле строится нетипизированное дерево. Затем средствами стандартной библиотеки компилятора оно транслируется в код языка OCaml.


\section{Эксперимент}
% !TeX spellcheck = ru_RU
% !TEX root = vkr.tex

\newcommand{\NA}{---}

В данном разделе приведены условия сравнительных экспериментов и их результаты.

\subsection{Цель эксперимента}

Целью эксперимента является сравнение производительности специализированных и исходных функций. 

\subsection{Условия эксперимента}

Для эксперимента были выбраны отношения: 
is\_even\footnote{Код отношения is\_even 
\url{https://github.com/IgorErin/SpecialKanren/blob/master/samples/is_even.ml} (Дата обращения 10.12.2023)}, sub\footnote{Код отношения sub : \url{https://github.com/IgorErin/SpecialKanren/blob/master/samples/sub.ml} (Дата обращения 10.12.2023)}, gcw\footnote{Код отношения gcw: \url{https://github.com/IgorErin/SpecialKanren/blob/master/samples/gcw.ml} (Дата обращения 10.12.2023)}, bridge\footnote{Исходный код отношения bridge \url{https://github.com/IgorErin/SpecialKanren/blob/master/samples/bridge.ml} (Дата обращения 10.12.2023)}.

Эксперименты проведены на машине, имеющей следующие характеристики: Ubuntu 20.4, AMD Ryzen 5 5500U, 4.4GHz, DDR4 16GB RAM. Для измерения был использован ocaml-benchmark\footnote{Репозиторий ocaml-benchmark: \url{https://github.com/Chris00/ocaml-benchmark} \\ (Дата обращение 10.12.2023)}.

\subsection{Результаты эксперимента}

Результаты измерений приведены в следующих таблицах, где \verb|x| и \verb|spec_x| обозначают соответственно результаты исполнения функции с конструктором \verb|x| и исполнение специализированной по этому конструктору функции. В столбце \verb|Частота| обозначено количество исполнений за секунду. Большее --- лучше. Во всех случаях отклонение составило менее 3\%, вследствие чего оно не приводится. Каждый эксперимент состоял из 30 замеров.

\begin{table}[H]
\begin{center}
\caption{Результаты измерений is\_even при получении первых ста ответов.} 
    \newcolumntype{P}[1]{>{\centering\arraybackslash}p{#1}}
    \begin{tabular}{|l|C{2cm}|C{2.7cm}|C{2.7cm}|}
    \hline
    Название & Частота & false & spec\_false  \\
    \hline
    \hline
    \rowcolor{black!10} false      &  485  & -- & -30\% \\
    \rowcolor{black!2}  spec\_false &  695   & 43\% & -- \\
    \hline
    \end{tabular}
    \label{evenfalse}
\end{center}
\end{table}

\begin{table}[H]
\begin{center}
\caption{Результаты измерений is\_even при получении первых ста ответов.} 
    \newcolumntype{P}[1]{>{\centering\arraybackslash}p{#1}}
    \begin{tabular}{|l|C{2cm}|C{2.7cm}|C{2.7cm}|}
    \hline
    Название & Частота & true & spec\_true  \\
    \hline
    \hline
    \rowcolor{black!10} true      &  486  & -- & -29\% \\
    \rowcolor{black!2}  spec\_true &  688   & 42\% & -- \\
    \hline
    \end{tabular}
    \label{eventrue}
\end{center}
\end{table}

\begin{table}[H]
\begin{center}
\caption{Результаты измерений sub при получении первых двадцати пяти ответов.} 
    \newcolumntype{P}[1]{>{\centering\arraybackslash}p{#1}}
    \begin{tabular}{|l|C{2cm}|C{2.7cm}|C{2.7cm}|}
    \hline
    Название & Частота & none & spec\_none  \\
    \hline
    \hline
    \rowcolor{black!10} none      &  5598  & -- & -24\% \\
    \rowcolor{black!2}  spec\_none &  7329   & 31\% & -- \\
    \hline
    \end{tabular}
    \label{subnone}
\end{center}
\end{table}

\begin{table}[H]
\begin{center}
\caption{Результаты измерений sub при получении первых двадцати пяти ответов. } 
    \newcolumntype{P}[1]{>{\centering\arraybackslash}p{#1}}
    \begin{tabular}{|l|C{2cm}|C{2.7cm}|C{2.7cm}|}
    \hline
    Название & Частота & some & spec\_some  \\
    \hline
    \hline
    \rowcolor{black!10} some      &  1071  & -- & -4\% \\
    \rowcolor{black!2}  spec\_some &  1113   & 4\% & -- \\
    \hline
    \end{tabular}
    \label{subsome}
\end{center}
\end{table}

\begin{table}[H]
\begin{center}
\caption{Результаты измерений gcw при получение первых ста ответов.} 
    \newcolumntype{P}[1]{>{\centering\arraybackslash}p{#1}}
    \begin{tabular}{|l|C{2cm}|C{2.7cm}|C{2.7cm}|}
    \hline
    Нзавние & Частота &  spec\_false & false  \\
    \hline
    \hline
    \rowcolor{black!10} spec\_false &  366  & -- & -19\% \\
    \rowcolor{black!2}  false      &  454   & 24\% & -- \\
    \hline
    \end{tabular}
    \label{gcwfalse}
\end{center}
\end{table}

\begin{table}[H]
\begin{center}
\caption{Результаты измерений gcw при получении первых ста ответов.} 
    \newcolumntype{P}[1]{>{\centering\arraybackslash}p{#1}}
    \begin{tabular}{|l|C{2cm}|C{2.7cm}|C{2.7cm}|}
    \hline
    Название & Частота & true & spec\_true  \\
    \hline
    \hline
    \rowcolor{black!10} true      &  1.25 & -- & -52\% \\
    \rowcolor{black!2}  spec\_true &  2.61 & 108\% & -- \\
    \hline
    \end{tabular}
    \label{gcwtrue}
\end{center}
\end{table}

\begin{table}[H]
\begin{center}
\caption{Результаты измерений bridge при получении первых ста ответов.} 
    \newcolumntype{P}[1]{>{\centering\arraybackslash}p{#1}}
    \begin{tabular}{|l|C{2cm}|C{2.7cm}|C{2.7cm}|}
    \hline
    Название & Частота &  none & spec\_none  \\
    \hline
    \hline
    \rowcolor{black!10} none      &  421  & -- & -29\% \\
    \rowcolor{black!2}  spec\_none &  596   & 42\% & -- \\
    \hline
    \end{tabular}
    \label{brdigenone}
\end{center}
\end{table}

Ввиду продолжительности исполнения отношения bridge, приведена величина, обратная \verb|Частоте|. То есть в столбце \verb|Скорость| обозначено среднее количество секунд за исполнение. Меньшее --- лучше.

\begin{table}[H]
\begin{center}
\caption{Результаты измерений bridge при получении первого ответа.} 
    \newcolumntype{P}[1]{>{\centering\arraybackslash}p{#1}}
    \begin{tabular}{|l|C{2.3cm}|C{2.7cm}|C{2.7cm}|}
    \hline
    Нзавние & Скорость & some & spec\_some  \\
    \hline
    \hline
    \rowcolor{black!10} some      &  15.8 & -- & -43\% \\
    \rowcolor{black!2}  spec\_some &  9.03 & 75\% & -- \\
    \hline
    \end{tabular}
    \label{brdigesome}
\end{center}
\end{table}

\FloatBarrier

\subsection{Выводы}

Из приведенных результатов видно, что специализированные версии эффективнее всегда, за исключением эксперимента \ref{gcwfalse}, где проигрыш может быть объяснен большим количеством дублицированного кода вследствие использования ДНФ. Во многих конъюнктах создаются большое количество свежих имен, которые в исходной формуле аллоцировались единожды. Данный случай показывает необходимость слияния одинаковых частей конъюнктов, по крайней мере аллокации переменных.

\section*{Заключение}
% !TeX spellcheck = ru_RU
% !TEX root = vkr.tex

В ходе данной работы были выполнены следующие задачи.

\begin{itemize}
    \item Реализован простой специализатор реляционных программ.
    
    Код работы доступен в репозитории на сервисе GitHub\footnote{Репозиторий проекта OCanren: \url{https://github.com/IgorErin/SpecialKanren} \\
    (Дата обращения: 10.12.2023)}. Имя
    пользователя: IgorErin. 
    
    \item Произведено сравнение изменения производительности некоторых функций до специализации и после.
\end{itemize}


Даже такая простая специализация способна значительно улучшить время исполнения реляционных программ.

Далее предполагается:

\begin{itemize}
    \item Увеличить класс обрабатываемых программ. Добавить поддержку модулей, частичного применения и тому подобного.
    \item Улучшить качество генерируемого кода. Уменьшить количество дублицируемых конструкций, поддержать инлайнинг простых реляций.
\end{itemize}

После чего необходимо будет провести эксперимент на настоящих программных продуктах, активно использующих реляционное программирование, например~\cite{JGS}.

\setmonofont[Mapping=tex-text]{CMU Typewriter Text}
  \bibliographystyle{ugost2008ls}
  \bibliography{vkr}
\end{document}
